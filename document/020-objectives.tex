\chapter{Objectives and Hypotheses}
\label{sec:objectives_and_hypotheses}

% \section{Kernel summary}
% \label{sec:kernels}

\begin{table}[H]
    \caption{Summary of infinite neural network kernels with analytical expressions found in the literature.}
    \label{tab:kernels_summary}
    \begin{tabular}{lcccc}
        \toprule
        \textbf{Kernel} & \textbf{Distribution} & \textbf{Activation function} & \textbf{Reference} & \textbf{Equation} \\
        \midrule
        Arc-sine         & \textit{Gaussian} & \textit{erf}       & \cite{williamsComputationInfiniteNeural1998,frenayParameterinsensitiveKernelExtreme2011} & \ref{eq:kernel_asin} \\
        \addlinespace
        Arc-cosine $n=0$ & \textit{Gaussian} & \textit{heavyside} & \cite{choLargemarginClassificationInfinite2010,pandeyGoDeepWide2014}                     & \ref{eq:kernel_cho_n0}  \\
        Arc-cosine $n=1$ & \textit{Gaussian} & \textit{ReLu}      & \cite{choLargemarginClassificationInfinite2010,pandeyGoDeepWide2014}                     & \ref{eq:kernel_cho_n1}  \\
        Arc-cosine $n=2$ & \textit{Gaussian} & \textit{RePu}      & \cite{choLargemarginClassificationInfinite2010,pandeyGoDeepWide2014}                     & \ref{eq:kernel_cho_n2}  \\
        \bottomrule
    \end{tabular}
\end{table}


\Cref{tab:kernels_summary} gives an overview of the infinite neural network
kernels with known analytical expressions in the literature which will be
used in this thesis.
We have 4 kernels in total, all of them derived from infinite neural networks
with Gaussian priors but with different activation functions. All of them have a
single hyperparameter, $\sigma_w$, which is the standard deviation of the
Gaussian prior. Additionally, we can compare the \emph{normalized} and
\emph{non-normalized} versions of the kernels. This gives a potential total of
8 kernels to compare.

\section{Objectives}
We can state the following objectives for the thesis:
\begin{description}
    \item[O1] Reproduce the results obtained by \textcite{frenayParameterinsensitiveKernelExtreme2011}.
    \item[O2] Compare the performance of the kernels with different datasets and hyperparameters.
    \item[O3] Apply meta-learning to find the best kernel and hyperparameters for a given dataset.
    \item[O4] Keep the computational cost of the experiments low in order to be able to run them in a reasonable amount of time.
        \begin{itemize}
            \item The experiments should be able to run on a laptop.
            \item The experiments should be able to run in a reasonable amount of time (less than a day).
        \end{itemize}
    \item[O5] Make the experiments reproducible and the code available to the public. % TODO: add link to code, and Zenodo
        \begin{itemize}
            \item The code should be well documented.
            \item The code should be easy to use.
            \item All dependencies should be documented.
        \end{itemize}
\end{description}

The first are related to the results of our experiments and the last two are
constraints on how the experiments should be performed in order to ensure that
they are both reproducible and computationally feasible.

\section{Hypotheses}

% TODO: formalize all the hypothesis
We formulate the following hypotheses regrading the behaviour of the kernels
and their hyperparameters.
These hypotheses are based on the results obtained by \textcite{frenayParameterinsensitiveKernelExtreme2011}:

\begin{hypothesis}\label{hyp:threshold}
    Kernels originating from infinite neural networks will
    exhibit the same behaviour described by \textcite{frenayParameterinsensitiveKernelExtreme2011};
    where there is a threshold value of their hyperparameter past which
    the kernel prediction capability does not improve.
\end{hypothesis}

\begin{hypothesis}
    The value of the hyper-parameter at the threshold described in \ref{hyp:threshold}, will
    be the optimal (or close to optimal) value for the kernel.
\end{hypothesis}

\begin{hypothesis}
    Infinite neural network kernels are able to generalize to other datasets and obtain
    results comparable to the \emph{state-of-the-art}.
\end{hypothesis}

The aim is to design experiments that will allow us to test these hypotheses.
